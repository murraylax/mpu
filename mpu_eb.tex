\documentclass[12pt]{article}
\usepackage[T1]{fontenc}
\usepackage{calc}
\usepackage{setspace}
\usepackage{multicol}
\usepackage{fancyheadings}
 
\usepackage{graphicx}
\usepackage{color}
\usepackage{rotating}
\usepackage{harvard}
\usepackage{aer}
\usepackage{aertt}
\usepackage{verbatim}
\usepackage{array}
\usepackage{multirow}

\setlength{\voffset}{-0.25in}
\setlength{\topmargin}{0pt}
\setlength{\hoffset}{0pt}
\setlength{\oddsidemargin}{0pt}
\setlength{\headheight}{0pt}
\setlength{\headsep}{.4in}
\setlength{\marginparsep}{0pt}
\setlength{\marginparwidth}{0pt}
\setlength{\marginparpush}{0pt}
\setlength{\footskip}{.1in}
\setlength{\textwidth}{6.5in}
\setlength{\textheight}{9.25in}
\setlength{\parskip}{0pc}

\renewcommand{\baselinestretch}{1.0}

\newcommand{\bi}{\begin{itemize}}
\newcommand{\ei}{\end{itemize}}
\newcommand{\be}{\begin{enumerate}}
\newcommand{\ee}{\end{enumerate}}
\newcommand{\bd}{\begin{description}}
\newcommand{\ed}{\end{description}}
\newcommand{\prbf}[1]{\textbf{#1}}
\newcommand{\prit}[1]{\textit{#1}}
\newcommand{\beq}{\begin{equation}}
\newcommand{\eeq}{\end{equation}}
\newcommand{\bdm}{\begin{displaymath}}
\newcommand{\edm}{\end{displaymath}}
\newcommand{\script}[1]{\begin{cal}#1\end{cal}}
\newcommand{\citee}[1]{\citename{#1} (\citeyear{#1})}
\newcommand{\h}[1]{\hat{#1}}
\newcommand{\ds}{\displaystyle}
\newcommand{\normal}{\mathcal{N}}
\newcommand{\app}
{
\appendix
}

%\newcommand{\appsection}[1]
%{
%\let\oldthesection\thesection
%\renewcommand{\thesection}{Appendix \oldthesection}
%\section{#1}\let\thesection\oldthesection
%\renewcommand{\theequation}{\thesection\arabic{equation}}
%\setcounter{equation}{0}
%}

\newcommand{\appsection}[1]
{
\section{#1}
\renewcommand{\theequation}{\thesection\arabic{equation}}
\setcounter{equation}{0}
}


%\pagestyle{empty}
\pagestyle{fancyplain}
\lhead{}
\chead{Monetary Policy Uncertainty and the Macroeconomy}
\rhead{\thepage}
\lfoot{}
\cfoot{}
\rfoot{}

\begin{document}

\noindent ``Improving the public's understanding of the central bank's objectives and policy strategies reduces economic and financial uncertainty and thereby allows business and households to make more informed decisions.''
$\sim$ Ben S. Bernanke, Speech to the Cato Institute 25th Annual Monetary Conference, November 17, 2007.

\section{Introduction}
Bernanke demonstrates in this quote that the Federal Reserve recognizes the value in keeping the public informed about the conduct of monetary policy.  Even so, the Federal Reserve has a dual mandate to promote both employment and inflation stability and it does not explicitly communicate relative importance for each of these goals, and does not communicate an explicit long-run target for inflation as central banks from some other countries do.  One might argue the reason for being vague is to give monetary policy flexibility to address new short-run economic challenges while maintaining credibility to keep inflation at moderate levels in the long run.  However, the lack of complete communication concerning the conduct of monetary policy may create some uncertainty among market participants concerning short-run and long-run monetary policy actions.  The purpose of this paper is to measure the degree of monetary policy uncertainty in the U.S. economy over the last several decades, and examine the effect uncertainty has on levels of output growth, unemployment, inflation, and the volatility of these variables.

Many authors have found monetary policy transparency and credibility important are for macroeconomic stability.  For example, \citee{cecchetti_krause2002} find evidence for this from 60 central banks around the world.  \citee{cecchetti2006} find for 20 countries around the world that 80\% of the reduction in macroeconomic volatility since the early 1980s can be attributed to better monetary policy, and that credibility and transparency plays an important role.  \citee{bernanke1997} suggests that the decrease in macroeconomic volatility since the early 1980s in the United States was due in large part to an established, and therefore well understood, monetary policy that put its greatest emphasis on inflation targets.  \citee{cecchetti_ehrmann2002} similarly find evidence for countries across the world that central banks that have shifted focus to inflation stability, either explicitly or implicitly, have sucessfully limited both inflation and output volatility since the early 1980s.

All these papers suggest that monetary policy that is well understood by the public leads greater macroeconomic stability, and many attribute the slow down of macroeconomic volatility around the world since the early 1980s to precisely this.  A related literature examines the effect macroeconomic volatility has on levels of inflation and output growth, where volatility is used as a measure for economic uncertainty.  Examples from this literature include \citee{grier_perry2000}, \citee{fountas2001}, \citee{fountas2002}, \citee{grier2004}, \citee{fountas2006}, and \citee{fountas2007}.  All these papers use autoregressive heteroskedastic models, with varying complications, to establish measures of economic uncertainty.  While results sometimes depend on the specification of the model, most of the papers agree that higher inflation uncertainty has a negative impact on economic growth.  In a sense, the implication for monetary policy may agree with \citee{bernanke1997} in that successful inflation targeting can lead to better macroeconomic outcomes.

The above papers are limited in that they do not focus specifically on uncertainty concerning monetary policy, and they cannot separate heteroskedasticity and uncertainty, so as to determine the impact uncertainty has volatility.  The present paper takes a step in each of these directions.  Motivated by the literature on transparency and inflation targeting which suggests well-understood policy leads to desirable outcomes, the present work measures monetary policy uncertainty in the U.S. by measuring market participants perceptions of monetary policy.  Specifically, we suppose agents estimate a Taylor-like regression rule where the federal funds rate responds to inflation, output growth, and unemployment.  Since the Fed does not explicitly communicate the relative importance of inflation and employment stability, the target inflation rate, or how responsive the federal funds rate is to fluctuations in these variables, we argue monetary policy is transparent when its actions are predictable, based on estimates of the linear regression monetary policy rule using data available to agents at the time.  

We re-estimate the Taylor-like regression rule for each period in our sample, using only the data prior to this period which would realistically be available to market participants at the time.  Specifically, we use a constant-gain least squares learning algorithm in the style of \citee{evanshonka2001} which supposes agents give relatively more weight to more recent observations.  We use the root mean squared error from this regression as our measure of monetary policy uncertainty and report the evolution of agents' expectations and agents' levels of uncertainty over the sample period.  We then estimate the impact uncertainty has on levels of output growth, unemployment and inflation and the volatility of these variables.  We fail to find evidence that uncertainty affects the levels of these variables, but we find statistically significant evidence that higher uncertainty leads to greater volatility of output growth and unemployment.

\section{Estimating Monetary Policy}
\subsection{Data}
We use quarterly data on output growth, inflation, unemployment, and the federal funds rate from 1965:Q1 though 2010:Q2.  Output growth is measured using the annualized quarterly percentage growth rate in real GDP, and inflation is measured using the annualized quarterly percentage growth rate in the GDP deflator.  The data was obtained using the Federal Reserve Bank of St. Louis FRED database.

\subsection{Learning}
Agents learn how the Federal Reserve conducts monetary policy by estimating the following policy rule similar to Taylor (1993),
\beq \label{eq:taylor} r_t = \alpha_0 + \alpha_r r_t + \alpha_{\pi} \pi_{t-1} + \alpha_g g_{t-1} + \alpha_u u_{t-1} + \epsilon_t \eeq
which recognizes that the Fed may adjust the nominal federal funds rate rate ($r_t$) in response to past inflation ($\pi_{t-1}$), past growth rate of real GDP ($g_{t-1}$), and past unemployment ($u_{t-1}$).  \citee{taylor1999} and \citee{orphanides2003}, among others, have suggest the Taylor rule is a useful framework to use to understand the conduct of monetary policy.  \citee{mccallum1997} also suggests it is realistic to have a Taylor rule that has the federal funds rate decision responding to lagged values of variables instead of concurrent values as this best reflects the actual data available to policy makers at the time of a decision.  Using lagged values also makes a convenient equation for market participants to estimate using least-squares techniques as the explanatory variables are exogenous.  

At every time $t$ agents re-estimate equation (\ref{eq:taylor}) using past data up through period $t-1$.  Let $x_{\tau} = [1~ \pi_{\tau-1}~ g_{\tau-1}~ u_{\tau-1}]'$ denote the vector of explanatory variables used to predict $r_{\tau}$ and $\hat{\alpha}_t = [\hat{\alpha}_{0,t}~ \hat{\alpha}_{\pi,t}~  \hat{\alpha}_{g,t}~  \hat{\alpha}_{u,t}]'$ denote the time $t$ estimate for the regression coefficients.  If agents estimate equation (\ref{eq:taylor}) by ordinary least squares, then the estimates for the coefficient at time $t$ is given by,
\beq \label{eq:ols} \hat{\alpha}_t = \left( \frac{1}{t} \sum_{\tau=1}^{t} x_{t-\tau} x_{t-\tau}' \right)^{-1}  \left( \frac{1}{t} \sum_{\tau=1}^{t} x_{t-\tau}  r_{t-\tau} \right). \eeq
This can be conveniently re-written in the following recursive form,
\beq \label{eq:lncoef} \hat{\alpha}_t = \hat{\alpha}_{t-1} + \gamma_t  R_t^{-1} x_{t-1} \left(r_{t-1} - x_{t-1}'\hat{\alpha}_{t-1}\right) \eeq
\beq \label{eq:lnR} R_t = R_{t-1} + \gamma_t \left(x_{t-1} x_{t-1}' - R_{t-1}\right) \eeq
where $g_{t-1} = 1/t$ is called the learning gain and is equal to the weight given to the most recent observation.  This recursive representation nicely illustrates the manner in which expectations are adaptive.  The term in parentheses on the right hand side of equation (\ref{eq:lncoef}) is the error that was made forecasting $r_{t-1}$ using the previous period's estimate for coefficients.  The degree to which the current estimate for $\alpha_t$ is updated from the previous estimate $\alpha_{t-1}$ depends on the forecast error and the size of the learning gain, $\gamma_t$.  The larger is the error made with the previous estimate, the larger is the update.  The larger is the learning gain, the larger is the update.  Since the learning gain is the inverse of the sample size, it is large when the sample size is small.  When the sample size is small, adding a new observation has a relatively large impact on the estimated coefficients.  As time approaches infinity, the sample size approaches infinity and the learning gain approaches zero.  When there are a large number of observations, a new observation has a negligible effect on the estimates.
 
As time progresses with ordinary least squares, the learning algorithm converges on a set of coefficients, and uncertainty about how the Fed conducts monetary policy disappears.  Also, if market participants always use ordinary least squares, they never suspect that a structural change in monetary policy is possible.  If a structural change did occur, market participants would learn about it, but only very slowly.  Structural change or not, the weight put on new observations gets smaller and smaller and all observations from the beginning of time are given equal weight.  

There is strong evidence that structural changes in Taylor rule occurred at multiple times in U.S. history.  \citee{taylor1999}, \citee{cgg2000}, and \citee{orphanides2003}, among others, find statistical evidence that the Federal Reserve more heavily targeted inflation after Paul Volker's appointment as Fed Chairman in 1979.  Constant gain learning is an alternative framework where agents can learn about structural changes and learning dynamics do not disappear over time.  Constant gain learning simply replaces the learning gain, $\gamma_t$, with a constant value, $\gamma \in (0,1)$.  Repeated substitution of equations (\ref{eq:lncoef}) and (\ref{eq:lnR}) shows that constant gain learning is equivalent to the following weighted least squares estimator,
\beq \label{eq:wls} \hat{\alpha}_t = \left( (1-\gamma)  \sum_{\tau=1}^{t} \gamma^{\tau} x_{t-\tau} x_{t-\tau}' \right)^{-1}  \left( (1-\gamma)  \sum_{\tau=1}^{t} \gamma^{\tau} x_{t-\tau}  r_{t-\tau} \right). \eeq
Equation (\ref{eq:wls}) indicates the weight observations from $\tau$ periods in the past is equal to $(1-\gamma)\gamma^{\tau}$.  Since $\gamma \in (0,1)$, most recent observations are given the highest weight and the weights decline geometrically with time.  One may view this as a learning mechanism for agents that have a constant suspicion of structural change that is not directly observable.  Agents do not have a formal understanding of the size of changes that could occur or the probabilities for which they could occur, so they simply put the most weight on the observations which are most likely to reflect the current structure of the economy.

Computing the coefficients for constant gain learning using the recursive algorithm given in equations (\ref{eq:lncoef}) and (\ref{eq:lnR}) requires an initial condition for $\hat{\alpha}_0$ and $R_0$.  The sample period studied in the this paper runs from 1965:Q1 though 2010:Q2.  We use the ten years prior to this sample (1955:Q1 through 1964:Q4) to construct an estimate for the initial conditions for the learning process.  We estimate the Taylor rule given in equation (\ref{eq:taylor}) with ordinary least squares and use the estimated coefficients to initialize $\hat{\alpha}_0$ and the average of the outer product, $x_t x_t'$, to initialize $R_0$.

Figure \ref{fg:coefs} shows the path of the constant gain learning coefficients for a learning gain, $\gamma=0.02$, which is a value close to what is estimated in the literature (see for example, \citee{milani2007} and \citee{slobodyan_wouters_2008}).  Most evident from these graphs is that agents have learned about frequent structural changes in the Taylor rule regression relationship; the coefficients do not remain near a constant value over time.  The coefficient on the lagged federal funds rate (measuring federal funds rate persistence) towards the end of the sample is only recently very close to 1.0, but from the early 1990s through early 2000s this persistence parameter was lower (about 0.85) and for the 20 years before that even lower (about 0.8).  There were also substantial swings in the coefficient in the 1970s and early 1980s.  Most notably there was a large drop in perceived persistence in 1980 when then Fed Chairman Paul Volcker substantially raised interest rates to combat very high inflation.  

The coefficient on inflation shows that there was an increase in the response of the federal funds rate to lagged inflation following the early 1970s and a most drastic, yet temporary, increase in 1980, as the Fed suddenly fought hard against inflation.  Following this episode, the response of the federal funds rate to inflation has remained relatively high compared to the early 1970s, and this finding in consistent with \citee{taylor1999}, \citee{cgg2000}, and \citee{orphanides2003}, as previously cited.  However, we see in Figure \ref{fg:coefs} that this coefficient has fallen to pre-1970 levels since the monetary easing following the 2001 recession under then Fed Chairman Alan Greenspan and most recently during the 2008-2009 recession under Fed Chairman Ben Bernanke.

The coefficients on output growth and unemployment also exhibit significant changes, also coinciding with the run-up of inflation in the 1970s, the evident regime change as Volcker becomes Chairman, and the recessions taking place in 1991, 2001, and 2008.  A particularly interesting pair of structural changes to the coefficient on unemployment takes place in 1980 and again in 1990.  The coefficient starts out negative, implying the Fed lowers the federal funds rate in response to an increase in unemployment.  The sudden movement towards 0.0 in 1980 shows again that the Fed changed its emphasis completely away from concerns about unemployment (and towards inflation as we saw earlier).  However, this was not a permanent change as other papers have implied.  Rather we see another drop in the coefficient (larger negative numbers implying the Fed is putting greater emphasis on unemployment) occurring just before the 1991 recession and remaining at lower levels compared to the 1980s.  

\subsection{Uncertainty}
The results from previous subsection show that market participants may have learned about a number of monetary policy changes from the last several decades.  Even so, such changes do take time to learn about, and changes in the conduct of monetary policy can have the immediate effect of agents making inaccurate forecasts.  An environment of uncertainty results when recent actions of the Federal Reserve deviate from market participants' expectations.  Deviations of market expectations are captured by the residuals from market participants weighted least-squares estimates of the Taylor rule given in equation (\ref{eq:taylor}).  The larger are the average squared residuals from this regression, the larger will be the variance of the forecast for $r_{t+\tau}$, and the larger will be variance for forecasts for any variable that depends on expectations of future interest rates.  For a given value for $\gamma$, we use the following root (weighted) mean squared residuals  (RMSR) as a measure of the degree of uncertainty caused by recent unpredicted monetary policy actions,
\beq m_{\gamma,t} = \sqrt{ (1-\gamma) \ds \sum_{\tau=1}^{t} \gamma^{\tau} (r_{t-\tau} - x_{t-\tau}'\hat{\alpha}_{t-\tau} )^2}. \eeq  
 
Figure \ref{fg:uncertain} shows the evolution of uncertainty over the sample period.  Despite market participants being able to learn about changes in monetary policy, as discussed in the last subsection, there have been some notable changes in uncertainty.  The run-up of inflation in the 1970s were also accompanied with a run-up of uncertainty.  As we saw earlier, agents were able to learn about, at least partially, Volker's significant changes in policy, but this period also marks the most significant increases in monetary policy uncertainty.  This high level of uncertainty continued until about 1984.  Since then, relative to the 1970s and early 1980s, uncertainty about monetary policy has been relatively small, but there still have been frequent jumps in the RMSR to 0.1, or 10 basis points, and towards the end of the sample the RMSR jumped to 20 basis points, the same time as the Fed decreased the federal funds rate to an historical low.
 
\section{Macroeconomic Impact}
In the previous section we quantified the degree of uncertainty among market participants concerning the conduct of monetary policy.  We now turn to estimating the impact this uncertainty has on the macroeconomy.  Specifically, we are interested in determining whether uncertainty adversely affects output growth, inflation, unemployment, or the volatility of these variables.  We answer this question in the context of reduced form vector autoregression (VAR) model with autoregressive conditional heteroskedastic (ARCH) shocks.  The VAR specification is general enough to allow for interactions of output growth, inflation, and unemployment as might be specified by a dynamic general equilibrium model or more simply-stated macroeconomic theory.  The ARCH shocks are added to allow for exogenous time-varying macroeconomic volatility.  

\subsection{Impact on Levels}
Consider the following augmented first order VAR(1),
\beq \label{eq:var} \begin{array}{cc}
g_t = & \beta_{g,0} + \beta_{g,g} g_{t-1} + \beta_{g,\pi} \pi_{t-1} + \beta_{g,u} u_{t-1} + \beta_{g,r} r_{t-1} + \beta_{g,\delta} + \lambda_{g} m_{\gamma,t}+ \nu_{g,t} \\ [0.5pc]
\pi_t = & \beta_{\pi,0} + \beta_{\pi,g} g_{t-1} + \beta_{\pi,\pi} \pi_{t-1} + \beta_{\pi,u} u_{t-1} + \beta_{\pi,r} r_{t-1} + \beta_{\pi,\delta} + \lambda_{\pi} m_{\gamma,t}+ \nu_{\pi,t} \\ [0.5pc]
u_t = & \beta_{u,0} + \beta_{u,g} g_{t-1} + \beta_{u,\pi} \pi_{t-1} + \beta_{u,u} u_{t-1} + \beta_{u,r} r_{t-1} + \beta_{u,\delta} + \lambda_{u} m_{\gamma,t}+ \nu_{u,t}, \\ 
\end{array} \eeq
where each of the stochastic shock terms, $\nu_{t}$, has a zero mean and possibly evolving variance, $\eta_{t}^2$, which is discussed in the next subsection.  Besides this, the standard VAR is augmented in another two ways.  First, we include the lagged interest rate, $r_{t-1}$, to allow monetary policy to influence these variables.  More significantly, we include the measure of monetary policy uncertainty, $m_{\gamma,t}$, as an explanatory variable to measure the impact uncertainty has on the levels for output growth, inflation, unemployment.  We saw in the previous section that this measure depends on a calibration for the learning gain.  In this paper, we consider the following three learning gains which are close to values found in the literature, $\gamma \in \{0.01, 0.02, 0.05\}$.

The estimation results are given in Table \ref{tb:var}.  The results show that the measure of market participants' uncertainty regarding monetary policy, $m_{\gamma,t}$, is not statistically significant in any of the regressions.   Therefore we fail to find evidence that this type of uncertainty affects levels of output growth, inflation, or unemployment.  The most significant explanatory variable in each regression is each variable's own lag, indicating these variables have significant degrees of persistence.  Besides this, we have two sets of statistically significant results that help describe the dynamics of the data.  We find that lagged output growth is significantly negatively related to unemployment, which is indicative of changes in unemployment lagging behind changes in the business cycle.  Also, lagged unemployment is significantly positively related to output growth, possibly as indication of ``V-shaped'' recession recoveries where periods of high unemployment at the end of a recession are followed by high levels of growth during the recovery.

\subsection{Impact on Macroeconomic Volatility}
The previous subsection found that monetary policy uncertainty does not affect \textit{levels} of output growth, inflation, and unemployment, but that does not rule out the possibility that monetary policy uncertainty affects the \textit{volatility} of the macroeconomy.  To test this possibility, we allow the variances of the stochastic shock terms in the VAR described above to evolve over time.  We model this with a relatively simple first order ARCH which allows the variances to evolve exogenously over time, but we augment the model to allow monetary policy uncertainty to also affect macroeconomic volatility.  We estimate the following models,
\beq \label{eq:arch} \begin{array}{cc}
\eta_{g,t}^2 = & \theta_{0,g} + \theta_{g,g} \eta_{g,t-1}^2 + \mu_g m_{\gamma,t}^2 + \upsilon_{g,t} \\ [0.5pc]
\eta_{\pi,t}^2 = & \theta_{0,\pi} + \theta_{\pi,g} \eta_{\pi,t-1}^2 + \mu_\pi m_{\gamma,t}^2 + \upsilon_{\pi,t} \\ [0.5pc]
\eta_{u,t}^2 = & \theta_{0,u} + \theta_{u,g} \eta_{u,t-1}^2 + \mu_u m_{\gamma,t}^2 + \upsilon_{u,t} \\ 
\end{array}
\eeq
where $\eta_t^2$ in each equation is the time $t$ variance for the stochastic shock from the previous VAR model, and $\upsilon_{t}$ in each equation is independently and identically distributed.  

The estimation results are given in Table \ref{tb:arch}.  The results show monetary policy uncertainty (as measured by $m_{\gamma,t}^2$) significantly explains the changing volatility of unemployment and output growth, but not inflation.  That is, higher monetary policy uncertainty leads to less stability in output growth and unemployment.

\section{Conclusion}
The Federal Reserve has a dual mandate to promote stability in employment and inflation and to maintain flexibility, it does not communicate precise targets for each nor the degree to which the federal funds rate will be adjusted in response to each.  Market participants decisions often depend on expectations for variables that depend on expectations for monetary policy.  We use a Taylor rule regression equation and a constant gain learning model to compute market participants estimates for the conduct of monetary policy, and we develop a measure for the degree of uncertainty caused by unpredicted monetary policy changes by aggregating recent squared residuals.  We find evidence consistent with other literature that monetary policy (as described by a Taylor rule) has evolved over the last several decades, along with market participants perceptions of monetary policy.  Despite the ability for market participants to learn about monetary policy, changes in policy also coincide with increases in monetary policy uncertainty.  

We incorporate the measure of monetary policy uncertainty into a VAR(1) model of output growth, inflation, unemployment, and interest rates with ARCH(1) errors.  The VAR(1) results indicate there is insufficient evidence to conclude monetary policy uncertainty affects levels of output growth, inflation, or unemployment, but the ARCH(1) results do show that higher monetary policy uncertainty leads to greater volatility for output growth and unemployment.  The policy implications may be important to central bankers if new challenges call for new monetary policy prescriptions.  Changes in policy may lead to an environment of increased uncertainty, which we find creates less stability in output growth and unemployment.

\newpage
\nocite{*}
\bibliographystyle{econometrica}
\bibliography{mpu}
\newpage

\begin{sidewaystable}\caption{Vector Autoregression Results$^1$}\label{tb:var}
\begin{center}
\begin{tabular}{l|p{0.71in} p{0.71in} p{0.71in}|p{0.71in} p{0.71in} p{0.71in}|p{0.71in} p{0.71in} p{0.71in}}
 & \multicolumn{3}{c|}{Dependent Var: Inflation ($\pi_t$)} & \multicolumn{3}{c|}{Dependent Var: Unemployment ($u_t$)} &  \multicolumn{3}{c}{Dependent Var: Output Growth ($g_t$)} \\ \hline
 & $\gamma=0.01$ & $\gamma=0.02$ & $\gamma=0.05$ & $\gamma=0.01$ & $\gamma=0.02$ & $\gamma=0.05$& $\gamma=0.01$ & $\gamma=0.02$ & $\gamma=0.05$ \\ \hline
Constant & 0.347 (0.241)    &  0.338 (0.239)    &  0.351 (0.241)     & 0.302*** (0.102)    &  0.312*** (0.103) &    0.310*** (0.104)    & 0.300 (0.259)    &  0.296 (0.259)    &  0.306 (0.262)    \\
$r_{t-1}$ & 0.063** (0.027)    &  0.064** (0.026)    &  0.062** (0.026)     & 0.001 (0.011)    &  0.000 (0.011)    &  0.001 (0.011)     & -0.031 (0.026)    &  -0.032 (0.026)    &  -0.034 (0.026)    \\
$g_{t-1}$ & 0.433*** (0.117)    &  0.435*** (0.117)    &  0.433*** (0.117)    & 0.014 (0.050)    &  0.013 (0.050)    &  0.013 (0.050)     & -0.045 (0.119)    &  -0.044 (0.119)    &  -0.045 (0.120)    \\
$\pi_{t-1}$ & 0.041 (0.082)    &  0.040 (0.082)    &  0.041 (0.082)     & -0.211*** (0.033)    &  -0.211*** (0.033)     &  -0.212*** (0.034)    & 0.312*** (0.085)   &  0.313*** (0.085)   &  0.315*** (0.086)  \\
$u_{t-1}$ & -0.025 (0.024)    &  -0.024 (0.025)    &  -0.026 (0.025)     & 0.968*** (0.013)   &  0.967*** (0.013)   &  0.968*** (0.014)    & 0.084*** (0.033)   &  0.085** (0.033)   &  0.083** (0.034) \\
$m_{\gamma,t}$ & -0.010 (0.091)    &  -0.017 (0.088)    &  -0.005 (0.084)     & 0.068 (0.049)    &  0.070 (0.046)    &  0.060 (0.043)     & -0.076 (0.116)    &  -0.072 (0.112)    &  -0.055 (0.109)    \\ \hline
$R^2$ & 0.367 & 0.367 & 0.367 & 0.971 & 0.971 & 0.971 & 0.176 & 0.175 & 0.173 \\ 
N & 181 & 181 & 181 & 181 & 181 & 181 & 181 & 181 & 181 \\ \hline
\multicolumn{10}{p{8in}}{$^1$ Heteroskedastic robust standard errors in parentheses.\newline
* Significant at the 10\% level.  ** Significant at the 5\% level.  *** Significant at the 1\% level.}\\ 
\end{tabular}
\end{center}
\end{sidewaystable}


\begin{sidewaystable}\caption{ARCH Results$^1$}\label{tb:arch}
\begin{center}
\begin{tabular}{l|p{0.71in} p{0.71in} p{0.71in}|p{0.71in} p{0.71in} p{0.71in}|p{0.71in} p{0.71in} p{0.71in}}
 & \multicolumn{3}{c|}{Inflation Volatility ($\eta_{\pi,t}^2$)} & \multicolumn{3}{c|}{Unemployment Volatility ($\eta_{u,t}^2$)} &  \multicolumn{3}{c}{Output Growth Volatility ($\eta_{g,t}^2$)} \\ \hline
 & $\gamma=0.01$ & $\gamma=0.02$ & $\gamma=0.05$ & $\gamma=0.01$ & $\gamma=0.02$ & $\gamma=0.05$& $\gamma=0.01$ & $\gamma=0.02$ & $\gamma=0.05$ \\ \hline
Constant & 0.323*** (0.109) & 0.321*** (0.109) & 0.323*** (0.109) &  0.057*** (0.013) & 0.057*** (0.013) & 0.057*** (0.013) &  0.485*** (0.101) & 0.487*** (0.101) & 0.485*** (0.101) \\
$\eta_{t-1}^2$ & 0.208*** (0.073) & 0.207*** (0.073) & 0.208*** (0.073) &  0.074   (0.074) & 0.086   (0.074) & 0.104   (0.074) &  0.109   (0.074) & 0.112   (0.075) & 0.115   (0.074) \\
$m_{\gamma,t}^2$ & 0.019   (0.013) & 0.022   (0.015) & 0.019   (0.013) &  0.008*** (0.002) & 0.006*** (0.002) & 0.005*** (0.002) &  0.037** (0.014) & 0.033** (0.014) & 0.031** (0.012) \\ \hline
$R^2$ & 0.057 & 0.058 & 0.059 & 0.081 & 0.083 & 0.096 & 0.058 & 0.053 & 0.055 \\ 
N & 180 & 180 & 180 & 180 & 180 & 180 & 180 & 180 & 180 \\ \hline
\multicolumn{10}{p{8in}}{$^1$ Standard errors in parentheses.\newline
* Significant at the 10\% level.  ** Significant at the 5\% level.  *** Significant at the 1\% level.}\\ 
\end{tabular}
\end{center}
\end{sidewaystable}


\begin{figure}\caption{Estimated Regression Coefficients with Learning ($\gamma=0.02$)}\label{fg:coefs}
\begin{center}
\begin{tabular}{cc}
\includegraphics[scale=0.4]{coef_fedfunds.png} & \includegraphics[scale=0.4]{coef_inflation.png} \\ [1pc]
\includegraphics[scale=0.4]{coef_growth.png} & \includegraphics[scale=0.4]{coef_unemployment.png} \\
\end{tabular}
\end{center}
\end{figure}

\begin{figure}\caption{Uncertainty About Monetary Policy: Root (Weighted) Mean Squared Residuals}\label{fg:uncertain}
\begin{center}
\includegraphics[scale=0.5]{uncertainty.png}
\end{center}
\end{figure}


\end{document}


